\documentclass[diplomskirad]{fer}
\title{Dynamic fluid visualization using smoothed particle hydrodynamics method}
\naslov{Vizualizacija dinamike fluida metodom hidrodinamike zaglađujućih čestica}
\brojrada{542}
\mentor{Krešimir Trontl}
\author{Hrvoje Hemen}
\date{June, 2024}
\datum{lipanj, 2024.}
\begin{document}
    \maketitle
    \zadatak{HrvojeHemenZadatak.pdf}
    \begin{zahvale}
        Želim se zahvaliti mentoru Krešimiru Trontlu-
    \end{zahvale}
    \mainmatter
    \tableofcontents
% TEKST RADA
    \chapter{Uvod}\label{ch:uvod}

    \section{Cilj rada}\label{sec:cilj-rada}

    Cilj ovog rada bio je napraviti realnu simulaciju dinamike fluida.
    Korištena metoda bila je metoda hidrodinamike zaglađujućih čestica (SPH).
    Inspiracija za ovaj rad bio je jedan YouTube video Sebastiana Laguea koji govori o simulaciji vode u Unityju.

    \section{Ukratko o radu}\label{sec:ukratko-o-radu}

    U sklopu ovog rada obrađeno je sve potrebno za samostalnu izradu ovog rada uključujući i postavljanje razvojnog okruženja.

    Rad je pisan u c\# programskom jeziku u sklopu Unityja, te je za vizualizaciju korišten Unityjev dvodimenzionalni vizualizator.


    \chapter{Tehnologije}\label{ch:tehnologije}

    \section{c\#}\label{sec:c}



    \section{Unity}\label{sec:unity}



    \chapter{Teorijska podloga}\label{ch:teorijska-podloga}

    \section{Pristupi računalnoj simulaciji fluida}\label{sec:pristupi-racunalnoj-simulaciji-fluida}

    \section{SPH metoda}\label{sec:sph-metoda}


    \chapter{Programska implementacija}\label{ch:programska-implementacija}

    \section{Osnove Unity okruženja}\label{sec:osnove-unity-okruzenja}
    \section{Osnove Unity fizičkog simulatora}\label{sec:osnove-unity-fizickog-simulatora}
    \section{čestica}\label{sec:cestica}
    \section{gustoća}\label{sec:gustoca}
    \section{pritisak}\label{sec:pritisak}
    \section{viskoza}\label{sec:viskoza}
    \section{rezultantna sila}\label{sec:rezultantna-sila}


    \bibliography{literatura}
    \begin{sazetak}
        sažetak na hrvatskom
    \end{sazetak}
    \begin{kljucnerijeci}
        ključne riječi na hrvatskom
    \end{kljucnerijeci}
    \begin{abstract}
        abstract in English
    \end{abstract}
    \begin{keywords}
        keywords in English
    \end{keywords}
\end{document}