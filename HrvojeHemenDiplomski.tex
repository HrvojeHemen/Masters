\documentclass[diplomskirad]{fer}

\usepackage{booktabs}
\usepackage{listings}
\usepackage[outputdir=out]{minted}

\newcommand{\paragraphnewline}[1]{\paragraph{#1}\mbox{}\\}


\title{Dynamic fluid visualization using smoothed particle hydrodynamics method}
\naslov{Vizualizacija dinamike fluida metodom hidrodinamike zaglađujućih čestica}
\brojrada{542}
\mentor{Krešimir Trontl}
\author{Hrvoje Hemen}
\date{June, 2024}
\datum{lipanj, 2024.}
\begin{document}
    \maketitle
    \zadatak{HrvojeHemenZadatak.pdf}
    \begin{zahvale}
        Želim se zahvaliti mentoru Krešimiru Trontlu-
    \end{zahvale}
    \mainmatter
    \tableofcontents
% TEKST RADA
    \chapter{Uvod}\label{ch:uvod}

    \section{Cilj rada}\label{sec:cilj-rada}

    Cilj ovog rada bio je napraviti realnu simulaciju dinamike fluida.
    Korištena metoda bila je metoda hidrodinamike zaglađujućih čestica (SPH).
    Inspiracija za ovaj rad bio je jedan YouTube video Sebastiana Laguea koji govori o simulaciji vode u Unityju.

    \section{Ukratko o radu}\label{sec:ukratko-o-radu}

    U sklopu ovog rada obrađeno je sve potrebno za samostalnu izradu ovog rada uključujući i postavljanje razvojnog okruženja.

    Rad je pisan u c\# programskom jeziku u sklopu Unityja, te je za vizualizaciju korišten Unityjev dvodimenzionalni vizualizator.


    \chapter{Tehnologije}\label{ch:tehnologije}

    \section{C\#}\label{sec:c}

    \begin{figure}[H]
        \centering
        \includegraphics[scale=0.1]{images/c-sharp}
        \caption{
            C Sharp Logo \cite{cSharpLogo}
        }
        \label{fig:cSharpLogo}
    \end{figure}

    C\# je objektno orijentirani programski jezik visoke razine.
    Nastao je ranih 2000-ih zbog potrebe za objektno orijentiranim jezikom sintakse slične C jeziku.
    Najvažni ciljevi njegovog razvoja bili su jednostavnost, stroga tipiziranost,
    laka prijenosnost na različite operacijske sustave i mala potrošnja računalnih resursa.

    Sintaksa je vrlo slična Javinoj, jer svaka naredba treba završiti sa točka-zarezom \";\".
    Također, dijelovi koda omeđeni su vitičastim zagradama, koje razdvajaju kod u Klase i Metode.
    Važna razlika C\# i Jave je to što C\# omogućava preopterećenje osnovnih operacija, dakle možemo reći klasi da kada
    upotrijebimo znak plus onda radi nešto drugo, a ne matematičko dodavanje.


    \newpage
    \section{Unity}\label{sec:unity}

    \begin{figure}[H]
        \centering
        \includegraphics[scale=0.3]{images/unityLogo}
        \caption{
            Unity Logo \cite{unityLogo}
        }
        \label{fig:unityLogo}
    \end{figure}

    Unity je razvojno okruženje i pogonski sklop za igre s mogućnosti razvoja 2D, 2.5D i 3D igara.
    Unity je nastao 2005 godine i od tada se kontinuirano raste i zauzima sve veći dio tržišta.

    Velika prednost Unityja nad drugim sličnim produktima je pristupačnost i opsežna dokumentacija.
    Pošto Unity omogućava razvoj za mobitele, Desktop platforme, web platforme, konzole te platforme virtualne realnosti,
    vrlo je jednostavno istu igru napraviti za više platformi.

    Postoji više licenci, među kojima postoji i besplatna razina.
    Ona omogućava novim programerima igara ulazak u taj svijet, te oni besplatno mogu vidjeti je li to za njih.



    \chapter{Teorijska podloga}\label{ch:teorijska-podloga}

    \section{Pristupi računalnoj simulaciji fluida}\label{sec:pristupi-racunalnoj-simulaciji-fluida}

    Kada pričamo o simulaciji fluida, najčešće mislimo na SPH metodu koju ovaj rad obrađuje, no postoji još mnogo
    različitih pristupa simulaciji fluida.
    Većina ih koristi čestice te simulira fluid nad njima, no neke koriste polja te pomoću njega računaju vizualiziraju čestice.

    \subsection{Simulacije bazirane na česticama}\label{subsec:simulacije-bazirane-na-cesticama}

    Simulacije bazirane na česticama su najintuitivnije.
    Svaka čestica predstavlja jednu česticu vode, te sadrži njena svojstva poput mase, brzine i vektora smjera kretanja.
    Kasnije se izračunavaju među-čestične sile poput tlaka i gustoće te se pomoću njih određuju nova svojstva čestica u
    sljedećem koraku simulacije te se to ponavlja.

    Najpoznatije simulacije ovog tipa su SPH koje će biti obrađeno zasebno, te DEM - Metoda diskretnih elemenata.

    DEM\cite{DEMmethod} metoda je većinski korištena za simulaciju građevinskog materijala poput piljevine ili pijeska.
    Ona uzima u obzir međučestične sile poput trenja, elastičnosti, i stavlja velik naglasak na Newtonove zakone.
    Najčešće se koristi u rudarskom inžinjerstvu, no postoje i primjene u farmaceutskoj industriji.

    \newpage
    \subsection{Simulacije bazirane na 2D polju}\label{subsec:simulacije-bazirane-na-2d-polju}

    Simulacije bazirane na 2D polju ne koriste čestice kao prijašnje metode, nego koriste 2D polje, u koje spremaju
    svojstva čestica koje bi se nalazile u ćeliji tog polja.
    One se kao i ostale simulacije izvode korak po korak.
    Ako želimo veliku simulaciju, njena složenost raste kvadratno, pa ove simulacije nisu prigodne za velike površine,
    baš zbog te velike računske složenosti.

    \begin{figure}[H]
        \centering
        \includegraphics[scale=0.5]{images/gridBasedParticleBased}
        \caption{
            Grid based, Particle based simulation \cite{gridBasedParticleBased}
        }
        \label{fig:gridBasedParticleBased}
    \end{figure}

    \section{SPH metoda}\label{sec:sph-metoda}

    \subsection{Općenito o metodi}\label{subsec:opcenito-o-metodi}

    SPH\cite{SPHmethod} metoda je računska metoda koja se koristi za simulaciju fluida.
    Ona je simulacija bazirana na česticama, te njihova svojstva koristi za računanje istih iz koraka u korak.
    Ona ne zahtjeva 2D polje, što je velika prednost, jer može simulirati fluid u složenim i nepravilnim prostorima.
    Njena složenost gledana s obzirom na broj čestica je puno manja nego složenost simulacija koje koriste 2D polje kada je gustoća jednaka.

    \newpage
    \subsection{Koraci simulacije}\label{subsec:koraci-simulacije}

    \paragraphnewline{Inicijalizacija simulacije}
    Na samom početku simulacije potrebno je definirati parametre simulacije poput jačine viskoze, međučestično odbijanje i slično.

    \paragraphnewline{Traženje susjeda}

    Prvi ``pravi`` korak simulacije je traženje susjeda.
    Vrlo je važno znati susjede kako bi se na čestice mogle primjeniti sile koje ovise o udaljenosti od drugih čestica poput viskoznosti i pritiska.
    Najčešći pristup traženju susjeda su pretraga pomoću 2D polja, u kojem svaku česticu na početku simulacije spremimo u polje, te gledamo bliskost čestica koje su i bliskim poljima radi bržeg iteriranja.

    \begin{figure}[H]
        \centering
        \includegraphics[scale=0.3]{images/Uniform-grid-searching-method}
        \caption{
            Uniform grid searching method \cite{uniformGridSearchingMethod}
        }
        \label{fig:uniformGridSearchingMethod}
    \end{figure}

    \newpage
    Drugi česti pristup ovome je pristup pomoću stabla, gdje se čestice spremaju u stablastu strukturu podataka i susjedi se lako traže spuštajući se niz nju.
    Susjedi se također mogu tražiti tako da se za svaku česticu ispituje udaljenost od svake druge, no složenost toga je kvadratna, te je jednostavno previše spora.

    \begin{figure}[H]
        \centering
        \includegraphics[scale=0.3]{images/kdtree}
        \caption{
            K-D stablo \cite{kdTree}
        }
        \label{fig:kdTree}
    \end{figure}

    \paragraphnewline{Računanje i primjena međučestičnih sila}

    Nakon pronalaska susjeda slijedi korak u kojem se računaju međučestične sile.
    Ovo je najvažniji korak jer upravo ove sile su te koje tjeraju čestice da se gibaju poput vode.
    Čestice ne smiju biti jedne u drugima, ali se svejedno moraju kretati zajedno i više čestica se mora ponašati složno.
    Najbitnije su sile sila pritiska, i sila viskoze.

    Prvo se računa gustoća, a zatim pomoću nje pritisak i onda nakon viskoza.

    Pritisak je bitan jer on tjera čestice jedne iz drugih kako nebi zapele jedna u drugu.
    \begin{figure}[H]
        \centering
        \includegraphics[scale=1]{images/pressureForce}
        \caption{
            Sile pritiska
        }
        \label{fig:pressureForce}
    \end{figure}

    Viskoza je suprotna pritisku, ona želi držati čestice na okupu tako da vuće čestice jedne prema drugima kako bi se micale u jednoj velikoj nakupini.
    Što je viskoza jača, fluid koji simuliramo bit će ``tvrđi`` i sve više ličiti na neku krutinu, jer će unutarnja sila biti toliko jaka da će se vanjske sile
    poput gravitacije moći zanemariti.
    \begin{figure}[H]
        \centering
        \includegraphics[scale=1]{images/viscoseForce}
        \caption{
            Sile viskoze
        }
        \label{fig:viscoseForce}
    \end{figure}



    % TODO dodat slike tih sila
    \paragraphnewline{Upravljanje rubnim uvjetima i sudarima}


    \chapter{Programska implementacija}\label{ch:programska-implementacija}

    \section{Osnove Unity okruženja}\label{sec:osnove-unity-okruzenja}
    \section{Osnove Unity fizičkog simulatora}\label{sec:osnove-unity-fizickog-simulatora}
    \section{Čestica}\label{sec:cestica}
    \section{Gustoća}\label{sec:gustoca}
    \section{Pritisak}\label{sec:pritisak}
    \section{Viskoza}\label{sec:viskoza}
    \section{Rezultantna sila}\label{sec:rezultantna-sila}


    \bibliography{literatura}
    \begin{sazetak}
        sažetak na hrvatskom
    \end{sazetak}
    \begin{kljucnerijeci}
        ključne riječi na hrvatskom
    \end{kljucnerijeci}
    \begin{abstract}
        abstract in English
    \end{abstract}
    \begin{keywords}
        keywords in English
    \end{keywords}
\end{document}